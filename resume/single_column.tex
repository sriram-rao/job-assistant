%%%%%%%%%%%%%%%%%
% This is an sample CV template created using altacv.cls
% (v1.7.2, 28 Aug 2024) written by LianTze Lim (liantze@gmail.com), based on the
% CV created by BusinessInsider at http://www.businessinsider.my/a-sample-resume-for-marissa-mayer-2016-7/?r=US&IR=T
%
%% It may be distributed and/or modified under the
%% conditions of the LaTeX Project Public License, either version 1.3
%% of this license or (at your option) any later version.
%% The latest version of this license is in
%%    http://www.latex-project.org/lppl.txt
%% and version 1.3 or later is part of all distributions of LaTeX
%% version 2003/12/01 or later.
%%%%%%%%%%%%%%%%

%% Use the "normalphoto" option if you want a normal photo instead of cropped to a circle
% \documentclass[10pt,a4paper,normalphoto]{altacv}

\documentclass[10pt,a4paper,ragged2e,withhyper]{altacv}
%% AltaCV uses the fontawesome5 and simpleicons packages.
%% See http://texdoc.net/pkg/fontawesome5 and http://texdoc.net/pkg/simpleicons for full list of symbols.

% Change the page layout if you need to
\geometry{left=1.25cm,right=1.25cm,top=1.25cm,bottom=1.25cm,columnsep=1cm}
% The paracol package lets you typeset columns of text in parallel
\usepackage{paracol}
\usepackage{multicol}

\newcommand{\greeting}{Hello} % The greeting to use (e.g. "Dear")
\newcommand{\recipientfirstname}{Hiring Manager}
\newcommand{\recipientlastname}{}

\newcommand{\company}{Microsoft AI} % The company (e.g. "Microsoft")
\newcommand{\fullcompany}{\company}

\newcommand{\city}{Mountain View, } % The company's city (e.g. "Redmond")
\newcommand{\state}{CA} % The company's state prefix (e.g. "WA")
\newcommand{\country}{} % The company's country (e.g. "India")
% \newcommand{\zip}{98052} % The company's zip code (e.g. "98052")

\newcommand{\mytitle}{Software Engineer} 
\newcommand{\jobtitle}{} % Your title (e.g. "Applicant")

\newcommand{\myindustryname}{large-scale systems}

\newcommand{\product}{}
% Personal information
\newcommand{\myfirstname}{Sriram}
\newcommand{\mylastname}{Rao}
\newcommand{\myname}{\myfirstname~\mylastname} 
\newcommand{\mywebsite}{sriramrao.com} 
\newcommand{\myemail}{reach@sriramrao.com} % Your email (e.g. "me@mydomain.com")
\newcommand{\mylinkedin}{sriram-rao} 
\newcommand{\phone}{+1 949 (560) 3250} % Your phone number (e.g. "123.456.7890")
\newcommand{\mylocation}{San Jose, CA} % Your general location (e.g. "PNW, USA")

% =========== Rarely Used=========== 
\newcommand{\recipient}{\recipientfirstname~\recipientlastname} % The recipient (e.g. "Hiring Manager")
\newcommand{\recipienttitle}{} % The recipient (e.g. "Hiring Manager")
\newcommand{\closer}{Yours truly} 
% Change the font if you want to, depending on whether
% you're using pdflatex or xelatex/lualatex
% WHEN COMPILING WITH XELATEX PLEASE USE
% xelatex -shell-escape -output-driver="xdvipdfmx -z 0" mmayer.tex
\iftutex
% \setmainfont{Lato}
    % \setmainfont{Fira Sans}
    \setmainfont{IBM Plex Sans}
    % \setmainfont{Source Sans Pro}
\else
  % If using pdflatex:
  \usepackage[default]{mathpazo}
\fi

\definecolor{deepcharcoal}{HTML}{1A1A1A}
\definecolor{charcoal}{HTML}{2E2E2E}
\definecolor{softblack}{HTML}{333333}
\definecolor{mediumgrey}{HTML}{777777} 
\definecolor{softgrey}{HTML}{B0B0B0} 

\definecolor{VividPurple}{HTML}{3E0097}
\definecolor{lavender}{HTML}{6A5ACD}
\definecolor{softindigo}{HTML}{3F51B5}
\definecolor{slateblue}{HTML}{6478A0}
\definecolor{teal}{HTML}{008080} 
\definecolor{softteal}{HTML}{4DB6AC} 
\definecolor{forestgreen}{HTML}{2E7D32}
\definecolor{burntumber}{HTML}{6E2C00}
\definecolor{redcurrant}{HTML}{A03D41}

\colorlet{name}{black}
\colorlet{tagline}{mediumgrey}
\colorlet{heading}{charcoal}
\colorlet{headingrule}{softgrey}
% \colorlet{subheading}{PastelRed}
\colorlet{emphasis}{softblack}
\colorlet{accent}{mediumgrey}
\colorlet{body}{softblack}

% Change some fonts, if necessary
% \renewcommand{\namefont}{\Huge\rmfamily\bfseries}
% \renewcommand{\personalinfofont}{\footnotesize}
% \renewcommand{\cvsectionfont}{\LARGE\rmfamily\bfseries}
% \renewcommand{\cvsubsectionfont}{\large\bfseries}

% Change the bullets for itemize and rating marker
% for \cvskill if you want to
\renewcommand{\cvItemMarker}{{\small\textbullet}}
\renewcommand{\cvRatingMarker}{\faCircle}
% ...and the markers for the date/location for \cvevent
% \renewcommand{\cvDateMarker}{\faCalendar*[regular]}
% \renewcommand{\cvLocationMarker}{\faMapMarker*}


% If your CV/résumé is in a language other than English,
% then you probably want to change these so that when you
% copy-paste from the PDF or run pdftotext, the location
% and date marker icons for \cvevent will paste as correct
% translations. For example Spanish:
% \renewcommand{\locationname}{Ubicación}
% \renewcommand{\datename}{Fecha}


%% Use (and optionally edit if necessary) this .tex if you
%% want to use an author-year reference style like APA(6)
%% for your publication list
% \input{pubs-authoryear.tex}

%% Use (and optionally edit if necessary) this .tex if you
%% want an originally numerical reference style like IEEE
%% for your publication list
\input{pubs-num.tex}

%% sample.bib contains your publications
\addbibresource{sample.bib}

\begin{document}

\newdimen\origiwspc%
\newdimen\origiwstr%
\origiwspc=\fontdimen2\font% original inter word space
\origiwstr=\fontdimen3\font% original inter word stretch

\name{\myname}
\tagline{\mytagline}
\personalinfo{%
  % Not all of these are required!
  % You can add your own with \printinfo{symbol}{detail}
  \phone{\myphone}
  \email{\myemail}
  \website{\mywebsite}
  % \location{\mylocation}
  \linkedin{\mylinkedin}
  \github{\mylinkedin} % I'm just making this up though.
%   \orcid{0000-0000-0000-0000} % Obviously making this up too.
  %% You can add your own arbitrary detail with
  %% \printinfo{symbol}{detail}[optional hyperlink prefix]
  % \printinfo{\faPaw}{Hey ho!}
  %% Or you can declare your own field with
  %% \NewInfoFiled{fieldname}{symbol}[optional hyperlink prefix] and use it:
  % \NewInfoField{gitlab}{\faGitlab}[https://gitlab.com/]
  % \gitlab{your_id}
	%%
  %% For services and platforms like Mastodon where there isn't a
  %% straightforward relation between the user ID/nickname and the hyperlink,
  %% you can use \printinfo directly e.g.
  % \printinfo{\faMastodon}{@username@instace}[https://instance.url/@username]
  %% But if you absolutely want to create new dedicated info fields for
  %% such platforms, then use \NewInfoField* with a star:
  % \NewInfoField*{mastodon}{\faMastodon}
  %% then you can use \mastodon, with TWO arguments where the 2nd argument is
  %% the full hyperlink.
  % \mastodon{@username@instance}{https://instance.url/@username}
}

\makecvheader

%% Depending on your tastes, you may want to make fonts of itemize environments slightly smaller
\AtBeginEnvironment{itemize}{\small}

%% Set the left/right column width ratio to 6:4.
% \columnratio{0.6}

% Start a 2-column paracol. Both the left and right columns will automatically
% break across pages if things get too long.
% \begin{paracol}{2}
\RaggedRight

\vspace{2mm}
\cvsection{Skills}
\LaTeXraggedright
% \fontdimen2\font=\origiwspc% (original) inter word space
% \fontdimen3\font=0.1em% inter word stretch
\fontsize{9}{15}\selectfont%
\vspace{-2mm}
\hyphenchar\font=-1
\sloppy
% Don't overuse these \cvtag boxes — they're just eye-candies and not essential. If something doesn't fit on a single line, it probably works better as part of an itemized list (probably inlined itemized list), or just as a comma-separated list of strengths.

% The `ragged2e` document class option might cause automatic linebreaks between \cvtag to fail.
% Either remove the ragged2e option; or 
% add \LaTeXraggedright in the paragraph for these \cvtag
{
\begin{paracol}{2}
\cvtagsingle{Languages}\newline
\mylangs
% \vspace{1mm}

% \vspace{-2mm}

\switchcolumn
% \divider

\cvtagsingle{Technologies}\newline
\myskills
\end{paracol}
}

\vspace{-1mm}
\cvsection{Experience}
% \hfill \cvtag{Industry}
\cveventonecolumn{\titlegsr}{\workgsr}{\timegsr}{\placegsr}
\detailgsr
\divider

\cveventonecolumn{\titledremio}{\workdremio}{\timedremio}{\placedremio}
\detaildremio 
\divider

\cveventonecolumn{\titlemicrosoft}{\workmicrosoft}{\timemicrosoft}{\placemicrosoft}
\detailmicrosoft 
\divider

\cveventonecolumn{\titleinternmicrosoft}{\workinternmicrosoft}{\timeinternmicrosoft}{\placeinternmicrosoft}
\detailinternmicrosoft 
\divider

\medskip

\hfill \cvtag{Academia}

% \vspace{-5mm}\cvevent{\titledrexel}{\workdrexel}{\timedrexel}{\placedrexel}
% \detaildrexel
% \divider
\vspace{-5mm}
\cveventonecolumn{\titleta}{\workta}{\timeta}{\placeta}
\detailta

% \divider

% \printbibliography[heading=pubtype,title={\printinfo{\faFile*[regular]}{Journal Articles}}, type=article]

% \divider

% \printbibliography[heading=pubtype,title={\printinfo{\faUsers}{Conference Proceedings}},type=inproceedings]

%% Switch to the right column. This will now automatically move to the second
%% page if the content is too long.
% \switchcolumn



% \vspace{2mm}
\cvsection{Education}
\normalsize

\cvevent{\titlemscs}{\workmscs}{\timemscs%
    \hspace{1cm}{\cvLocationMarker~\placemscs}}{}%

\divider

\cvevent{\titlephd}{\workphd}{\timephd%
    \hspace{1cm}{\cvLocationMarker~\placephd}}{}%
\detailphd

\divider

\cvevent{\titlebtech}{\workbtech}{\timebtech\hspace{1cm}{\cvLocationMarker~\placebtech}}{}%

\bigskip

\cvsection{Publications}
\medskip
%% Specify your last name(s) and first name(s) as given in the .bib to automatically bold your own name in the publications list.
%% One caveat: You need to write \bibnamedelima where there's a space in your name for this to work properly; or write \bibnamedelimi if you use initials in the .bib
\mynames{Rao\bibnamedelima Sriram,
  Rao\bibnamedelima Sriram,
  Rao/Sriram,
  Rao\bibnamedelimi S.}
%% You can specify multiple names, especially if you have changed your name or if you need to highlight multiple authors.
%% MAKE SURE THERE IS NO SPACE AFTER THE FINAL NAME IN YOUR \mynames LIST
\nocite{*}

\setstretch{1.1}
\printbibliography[heading=none]
% \cvsection{References}

% % \cvref{name}{email}{mailing address}
% \cvref{Prof.\ Alpha Beta}{Institute}{a.beta@university.edu}
% {Address Line 1\\Address line 2}

% \divider

% \cvref{Prof.\ Gamma Delta}{Institute}{g.delta@university.edu}
% {Address Line 1\\Address line 2}

% \end{paracol}

\end{document}