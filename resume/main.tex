%%%%%%%%%%%%%%%%%
% This is an sample CV template created using altacv.cls
% (v1.7.2, 28 Aug 2024) written by LianTze Lim (liantze@gmail.com), based on the
% CV created by BusinessInsider at http://www.businessinsider.my/a-sample-resume-for-marissa-mayer-2016-7/?r=US&IR=T
%
%% It may be distributed and/or modified under the
%% conditions of the LaTeX Project Public License, either version 1.3
%% of this license or (at your option) any later version.
%% The latest version of this license is in
%%    http://www.latex-project.org/lppl.txt
%% and version 1.3 or later is part of all distributions of LaTeX
%% version 2003/12/01 or later.
%%%%%%%%%%%%%%%%

%% Use the "normalphoto" option if you want a normal photo instead of cropped to a circle
% \documentclass[10pt,a4paper,normalphoto]{altacv}

\documentclass[10pt,a4paper,ragged2e,withhyper]{altacv}
%% AltaCV uses the fontawesome5 and simpleicons packages.
%% See http://texdoc.net/pkg/fontawesome5 and http://texdoc.net/pkg/simpleicons for full list of symbols.

% Change the page layout if you need to
\geometry{left=1.25cm,right=1.25cm,top=1.25cm,bottom=1.25cm,columnsep=1cm}
% The paracol package lets you typeset columns of text in parallel
\usepackage{paracol}

\newcommand{\myfirstname}{Sriram}
\newcommand{\mylastname}{Rao}
\newcommand{\myname}{\myfirstname~\mylastname} 


\newcommand{\mytagline}{\hyphenchar\font=-1
Software engineer with industry and research experience in distributed data systems.
Proven track record
in developing efficient, resilient solutions
and collaborating across teams to drive software innovation.}
\newcommand{\mytitle}{Software Engineer} % Your title (e.g. "Applicant")


\newcommand{\myemail}{reach@sriramrao.com} 
\newcommand{\mylinkedin}{sriram-rao} 
\newcommand{\myphone}{+1 (949) 560-3250} 
\newcommand{\mylocation}{San Jose, CA} 
\newcommand{\mywebsite}{sriramrao.com}

\newcommand{\mylangs}{ 

Python, C$\#$, Java, 
C++, C, Ruby, Swift, Lisp, 
Prolog, SQL \newline

\textit{UI/UX}: HTML, CSS, 
TypeScript (\& JS), SwiftUI \newline

\textit{Automation}: Bash, Powershell, Lua \newline
}

\newcommand{\myskills}{%

    \textit{Databases:} Big Data, NoSQL, MongoDB, OLAP, PostgreSQL, Column-stores.

    \textit{Compute Platforms:} Spark, ETL, DAG, Query Engine, Apache Calcite, Iceberg, Trino. 
    
    \textit{Backend:} .NET, Spring, Flask, REST, SvelteKit, Microservices, AOP, Architecture, Caching. \newline
    
    \textit{Infra:} Docker, AWS, Azure, CI/CD, Custom IaC. \newline
}

\newcommand{\makework}[6]{
    \expandafter\def\csname work#1\endcsname{#2}
    \expandafter\def\csname title#1\endcsname{#3}
    \expandafter\def\csname time#1\endcsname{#4}
    \expandafter\def\csname place#1\endcsname{#5}
    \expandafter\def\csname detail#1\endcsname{#6}
}

\makework{dremio}
    {Dremio}
    {Software Engineer - PhD Intern}
    {Jun 2022 - Sep 2022}
    {Remote, CA}
    {
    \begin{itemize}
        \item Devised a proof-of-concept (POC) to progressively improve query response in data lakes. Familiarized with \textbf{Apache Calcite} and \textbf{Iceberg}.
        
        \item Improved row-count estimation in \textbf{query planning/optimisation} via accurate statistics observed in prior executions. 
        (LEO, \href{https://www.vldb.org/conf/2001/P019.pdf}{Markl, VLDB 2001}).%
    \end{itemize}
    }

\makework{microsoft}
    {Microsoft}
    {Software Engineer 2}
    {Jun 2016 - Sep 2020}
    {Bengaluru, India}
    {
    \begin{itemize}
        \item Rebuilt workflow manager used for Extract-Transform-Load (\textbf{ETL}) in 100+ pipelines, reducing time-to-deploy from 1h to < 5s. 
        \item Piloted \textbf{Spark} Streaming POC pipeline to compute the statistical significance of A/B tests 3x faster than existing batched method.
        \item Refactored cache config. system in API hosted on \textbf{Azure} using Aspect-Oriented Programming. Decreased config. code 5x, codebase size 300 lines. 
        \item Contributed to teammates' success with detailed input on 40+ design reviews, 100+ \textbf{code reviews}, and live issues on call.
    \end{itemize}
    }

\makework{internmicrosoft}
    {Microsoft}
    {Summer Intern}
    {May 2015 - Jun 2015}
    {Bengaluru, India}
    {
    \begin{itemize}
        \item Analyzed insert \& response times of 3 data stores under stress loads.
        (Azure Data Explorer/\textbf{Kusto}, \textbf{MongoDB}, \textbf{column-store}.)

        \item Concluded Kusto suited the log analysis use-case (response < 5s), column-store the aggregate-based queries (response < 1s). 

        \item Enabled migration from analytical (\textbf{OLAP}) cubes (instant response) to column-store (response < 1s). Cut ETL time from 10 days to 1 hour. 
    \end{itemize}
    }

\makework{gsr}
    {University of California, Irvine}
    {Graduate Student Researcher}
    {Sep 2020 - Sep 2024}
    {Irvine, CA}
    {
    \begin{itemize}
        \item Designed database (DB) plugin that balances latency \& costs,
        allocating \textbf{query-processing} resources to ongoing \& \textbf{ML}-forecast loads.
    
        \item Developed framework for implicit  \textbf{simulator} invocation by DB. 
        Showed ease of analysis by integrating \href{https://www.ready.noaa.gov/HYSPLIT.php}{HYSPLIT} into PostgreSQL. 
        
        \item Created a \textbf{pipeline execution} system for workflows defined as directed acyclic graphs of tasks. (On \href{https://github.com/sriram-rao/rush}{GitHub} as sample.) 
    \end{itemize} 
     } 

\makework{ta}
    {University of California, Irvine}
    {Teaching Assistant}
    {Sep 2020 - Dec 2024}
    {Irvine, CA}
    {
    \begin{itemize}
        \item Rated 4/5 in anonymous feedback from students, with appreciation for database expertise and \textbf{straightforward explanation}. 
    
        \item Collaborated with professors \& TAs on lecture slides, questions, assignments, discussion hours in courses with 200+ students. 
 
    \end{itemize} 
     } 

\makework{drexel}
    {Drexel University}
    {Volunteer Intern}
    {May 2025 - Current}
    {Remote, CA}
    {
    \begin{itemize}
        \item Building an \textbf{iOS app} to intuitively record and export a DBT mood journal. 
    \end{itemize}
    }

\makework{mscs}
    {University of California, Irvine}
    {MS in Computer Science}
    {Sep 2020 - Mar 2025}
    {Irvine, CA}
    {}

\makework{phd}
    {University of California, Irvine}
    {Graduate Work in the PhD Program}
    {Sep 2020 - Sep 2024}
    {Irvine, CA}
    {
    \begin{itemize}
        \item Advised by Prof. Sharad Mehrotra in data management: workload-aware pre-computation.%
    \end{itemize}
    }

\makework{btech}
    {National Institute of Technology, Karnataka}
    {B. Tech. in Computer Engineering}
    {Jul 2012 - Mar 2016}
    {Surathkal, India}
    {}
% Change the font if you want to, depending on whether
% you're using pdflatex or xelatex/lualatex
% WHEN COMPILING WITH XELATEX PLEASE USE
% xelatex -shell-escape -output-driver="xdvipdfmx -z 0" mmayer.tex
\iftutex
% \setmainfont{Lato}
    % \setmainfont{Fira Sans}
    \setmainfont{IBM Plex Sans}
    % \setmainfont{Source Sans Pro}
\else
  % If using pdflatex:
%    \renewcommand{\familydefault}{\sfdefault}
    \setmainfont{sans-serif}
\fi

\definecolor{deepcharcoal}{HTML}{1A1A1A}
\definecolor{charcoal}{HTML}{2E2E2E}
\definecolor{softblack}{HTML}{333333}
\definecolor{mediumgrey}{HTML}{777777} 
\definecolor{softgrey}{HTML}{B0B0B0} 

\definecolor{VividPurple}{HTML}{3E0097}
\definecolor{lavender}{HTML}{6A5ACD}
\definecolor{softindigo}{HTML}{3F51B5}
\definecolor{slateblue}{HTML}{6478A0}
\definecolor{teal}{HTML}{008080} 
\definecolor{softteal}{HTML}{4DB6AC} 
\definecolor{forestgreen}{HTML}{2E7D32}
\definecolor{burntumber}{HTML}{6E2C00}
\definecolor{redcurrant}{HTML}{A03D41}

\colorlet{name}{black}
\colorlet{tagline}{mediumgrey}
\colorlet{heading}{charcoal}
\colorlet{headingrule}{softgrey}
% \colorlet{subheading}{PastelRed}
\colorlet{emphasis}{softblack}
\colorlet{accent}{mediumgrey}
\colorlet{body}{softblack}

% Change some fonts, if necessary
% \renewcommand{\namefont}{\Huge\rmfamily\bfseries}
% \renewcommand{\personalinfofont}{\footnotesize}
% \renewcommand{\cvsectionfont}{\LARGE\rmfamily\bfseries}
% \renewcommand{\cvsubsectionfont}{\large\bfseries}

% Change the bullets for itemize and rating marker
% for \cvskill if you want to
\renewcommand{\cvItemMarker}{{\small\textbullet}}
\renewcommand{\cvRatingMarker}{\faCircle}
% ...and the markers for the date/location for \cvevent
% \renewcommand{\cvDateMarker}{\faCalendar*[regular]}
% \renewcommand{\cvLocationMarker}{\faMapMarker*}


% If your CV/résumé is in a language other than English,
% then you probably want to change these so that when you
% copy-paste from the PDF or run pdftotext, the location
% and date marker icons for \cvevent will paste as correct
% translations. For example Spanish:
% \renewcommand{\locationname}{Ubicación}
% \renewcommand{\datename}{Fecha}


%% Use (and optionally edit if necessary) this .tex if you
%% want to use an author-year reference style like APA(6)
%% for your publication list
% \usepackage[backend=biber,style=ieee,sorting=ydnt,defernumbers=true]{biblatex}
%% For removing numbering entirely when using a numeric style
\renewcommand{\bibfont}{\small} 
\setlength{\bibhang}{1.25em}
\DeclareFieldFormat{labelnumberwidth}{\makebox[\bibhang][l]{\itemmarker}}
\setlength{\biblabelsep}{0pt}
\defbibheading{pubtype}{\cvsubsection{#1}}
\renewcommand{\bibsetup}{\vspace*{-\baselineskip}}
\AtEveryBibitem{%
  \small\iffieldundef{doi}{}{\clearfield{url}}%
}


%% Use (and optionally edit if necessary) this .tex if you
%% want an originally numerical reference style like IEEE
%% for your publication list
\usepackage[backend=biber,style=ieee,sorting=ydnt,defernumbers=true]{biblatex}
%% For removing numbering entirely when using a numeric style
\renewcommand{\bibfont}{\small} 
\setlength{\bibhang}{1.25em}
\DeclareFieldFormat{labelnumberwidth}{\makebox[\bibhang][l]{\itemmarker}}
\setlength{\biblabelsep}{0pt}
\defbibheading{pubtype}{\cvsubsection{#1}}
\renewcommand{\bibsetup}{\vspace*{-\baselineskip}}
\AtEveryBibitem{%
  \iffieldundef{doi}{}{\clearfield{url}}%
}


%% sample.bib contains your publications
\addbibresource{sample.bib}

\begin{document}

\newdimen\origiwspc%
\newdimen\origiwstr%
\origiwspc=\fontdimen2\font% original inter word space
\origiwstr=\fontdimen3\font% original inter word stretch

\name{\myname}
\tagline{\mytagline}
\personalinfo{%
  % Not all of these are required!
  % You can add your own with \printinfo{symbol}{detail}
  \phone{\myphone}
  \email{\myemail}
  \website{\mywebsite}
  % \location{\mylocation}
  \linkedin{\mylinkedin}
  \github{\mylinkedin} % I'm just making this up though.
%   \orcid{0000-0000-0000-0000} % Obviously making this up too.
  %% You can add your own arbitrary detail with
  %% \printinfo{symbol}{detail}[optional hyperlink prefix]
  % \printinfo{\faPaw}{Hey ho!}
  %% Or you can declare your own field with
  %% \NewInfoFiled{fieldname}{symbol}[optional hyperlink prefix] and use it:
  % \NewInfoField{gitlab}{\faGitlab}[https://gitlab.com/]
  % \gitlab{your_id}
	%%
  %% For services and platforms like Mastodon where there isn't a
  %% straightforward relation between the user ID/nickname and the hyperlink,
  %% you can use \printinfo directly e.g.
  % \printinfo{\faMastodon}{@username@instace}[https://instance.url/@username]
  %% But if you absolutely want to create new dedicated info fields for
  %% such platforms, then use \NewInfoField* with a star:
  % \NewInfoField*{mastodon}{\faMastodon}
  %% then you can use \mastodon, with TWO arguments where the 2nd argument is
  %% the full hyperlink.
  % \mastodon{@username@instance}{https://instance.url/@username}
}

\makecvheader

%% Depending on your tastes, you may want to make fonts of itemize environments slightly smaller
\AtBeginEnvironment{itemize}{\small}

%% Set the left/right column width ratio to 6:4.
\columnratio{0.6}

% Start a 2-column paracol. Both the left and right columns will automatically
% break across pages if things get too long.
\begin{paracol}{2}
\RaggedRight

\vspace{-2mm}
\cvsection{Experience}

\vspace{0.6mm}
% \hfill \cvtag{Industry}
\cvevent{\titlegsr}{\workgsr}{\timegsr}{\placegsr}
\detailgsr
\divider

\cvevent{\titledremio}{\workdremio}{\timedremio}{\placedremio}
\detaildremio 
\divider

\cvevent{\titlemicrosoft}{\workmicrosoft}{\timemicrosoft}{\placemicrosoft}
\detailmicrosoft 
\divider

\cvevent{\titleinternmicrosoft}{\workinternmicrosoft}{\timeinternmicrosoft}{\placeinternmicrosoft}
\detailinternmicrosoft 
\divider

\medskip

\hfill \cvtag{Academia}

% \vspace{-5mm}\cvevent{\titledrexel}{\workdrexel}{\timedrexel}{\placedrexel}
% \detaildrexel
% \divider
\vspace{-5mm}
\cvevent{\titleta}{\workta}{\timeta}{\placeta}
\detailta

% \divider

% \printbibliography[heading=pubtype,title={\printinfo{\faFile*[regular]}{Journal Articles}}, type=article]

% \divider

% \printbibliography[heading=pubtype,title={\printinfo{\faUsers}{Conference Proceedings}},type=inproceedings]

%% Switch to the right column. This will now automatically move to the second
%% page if the content is too long.
\switchcolumn

\vspace{-1mm}
\cvsection{Skills}
\LaTeXraggedright
% \fontdimen2\font=\origiwspc% (original) inter word space
% \fontdimen3\font=0.1em% inter word stretch
\fontsize{9}{15}\selectfont%
\hyphenchar\font=-1
\sloppy
% Don't overuse these \cvtag boxes — they're just eye-candies and not essential. If something doesn't fit on a single line, it probably works better as part of an itemized list (probably inlined itemized list), or just as a comma-separated list of strengths.

% The `ragged2e` document class option might cause automatic linebreaks between \cvtag to fail.
% Either remove the ragged2e option; or 
% add \LaTeXraggedright in the paragraph for these \cvtag
{
\cvtag{Languages} 
\vspace{2mm}
\mylangs
\vspace{-2mm}

\divider

\cvtag{Technologies}~ \myskills
}

\vspace{2mm}
\cvsection{Education}
\normalsize

\cvevent{\titlemscs}{\workmscs}{\timemscs%
    \hspace{1cm}{\cvLocationMarker~\placemscs}}{}%

\divider

\cvevent{\titlephd}{\workphd}{\timephd%
    \hspace{1cm}{\cvLocationMarker~\placephd}}{}%
\detailphd

\divider

\cvevent{\titlebtech}{\workbtech}{\timebtech\hspace{1cm}{\cvLocationMarker~\placebtech}}{}%

\bigskip

\cvsection{Publications}
\medskip
%% Specify your last name(s) and first name(s) as given in the .bib to automatically bold your own name in the publications list.
%% One caveat: You need to write \bibnamedelima where there's a space in your name for this to work properly; or write \bibnamedelimi if you use initials in the .bib
\mynames{Rao\bibnamedelima Sriram,
  Rao\bibnamedelima Sriram,
  Rao/Sriram,
  Rao\bibnamedelimi S.}
%% You can specify multiple names, especially if you have changed your name or if you need to highlight multiple authors.
%% MAKE SURE THERE IS NO SPACE AFTER THE FINAL NAME IN YOUR \mynames LIST
\nocite{*}

\setstretch{1.1}
\printbibliography[heading=none]
% \cvsection{References}

% % \cvref{name}{email}{mailing address}
% \cvref{Prof.\ Alpha Beta}{Institute}{a.beta@university.edu}
% {Address Line 1 \newline Address line 2}

% \divider

% \cvref{Prof.\ Gamma Delta}{Institute}{g.delta@university.edu}
% {Address Line 1 \newline Address line 2}

\end{paracol}

\end{document}